\documentclass{article}
\usepackage[utf8]{inputenc}
\usepackage{amsmath}% http://ctan.org/pkg/amsmath

\title{CS2C Assignment 2}
\author{Angus Yip}
\date{April 24th, 2017}

\usepackage{natbib}
\usepackage{graphicx}

\begin{document}

\maketitle

\section*{2.1}

    The ascending order for the functions are: 
    
    \[ \frac{2}{N}, 37, \sqrt{N}, N, N\log{\log{N}}, N\log{N}, N\log{N^2}, \]
    \[ N(\log{N})^2, N^{1.5}, N^2, N^2\log{N}, N^3, 2^\frac{N}{2}, 2^N \]
    
    \noindent Furthermore, the functions \( N\log{N} \) and \( N\log{N^2} \) 
    grows at the same rate.

\section*{2.2}
    \renewcommand{\labelenumi}{\alph{enumi}}
    \begin{enumerate}
        \item true.
        \item false.
        \item false.
        \item false.
    \end{enumerate}

\section*{2.6}
    %\renewcommand{\labelenumi}{\alph{enumi}}
    \begin{enumerate}
        \item \( 2^{2^{N - 1}} \)
        \item 
            \begin{equation*} 
            \begin{split}
                2^{2^{N - 1}} & = D \\
                2^{N - 1}\log{2} & = \log{D}\\
                2^{N - 1} & = \frac{\log{D}}{\log{2}}\\
                (N - 1)\log{2} & = \log{\log{D}} - \log{\log{2}}\\
                N & = \frac{\log{\log{D}}}{\log{2}} - \frac{\log{\log{2}}}{\log{2}} + 1\\
                N & = O(\log{\log{D}})
            \end{split}
            \end{equation*}
    
    
    \end{enumerate}

\section*{2.7}
    \begin{enumerate}
        \renewcommand{\labelenumi}{\arabic{enumi}}
        \item 
            \begin{enumerate}
                \item
                \item 
                \item
            \end{enumerate}
        \item
            \begin{enumerate}
                \item 
                \item 
                \item
            \end{enumerate}
        \item 
            \begin{enumerate}
                \item
                \item 
                \item
            \end{enumerate}
        \item
            \begin{enumerate}
                \item 
                \item 
                \item
            \end{enumerate}
        \item 
            \begin{enumerate}
                \item
                \item 
                \item
            \end{enumerate}
        \item
            \begin{enumerate}
                \item 
                \item 
                \item
            \end{enumerate}
    
    \end{enumerate}
    

\section*{2.11}
    %\renewcommand{\labelenumi}{\alph{enumi}}
    \begin{enumerate}
        \item
            \begin{equation*} 
            \begin{split}
                \frac{500}{100} & = \frac{N}{.5}\\
                N & = 2.50 ms
            \end{split}
            \end{equation*}
        \item 
            \begin{equation*}
            \begin{split}
                \frac{500\log{500}}{100\log{100}} & = \frac{N}{.5}\\
                N & = 3.37 ms
            \end{split}
            \end{equation*}
        \item
            \begin{equation*} 
            \begin{split}
                \frac{500^2}{100^2} & = \frac{N}{.5}\\
                N & = 12.5 ms
            \end{split}
            \end{equation*}
        \item
            \begin{equation*} 
            \begin{split}
                \frac{500^3}{100^3} & = \frac{N}{.5}\\
                N & = 62.5 ms
            \end{split}
            \end{equation*}
    \end{enumerate}

\section*{2.13}
    %\renewcommand{\labelenumi}{\alph{enumi}}
    \begin{enumerate}
        \item
            \( \sum_{i=0}^{N} a_{i}x^i \) takes \(i\) operations to compute.
            \begin{equation*} 
            \begin{split}
                & \sum_{i=0}^{N} i + N\\
                = &\, \frac{N(N + 1)}{2} + N\\
                = &\, O(N^2)
            \end{split}
            \end{equation*}
        \item 
            From section 2.4.4, the new algorithm takes \( \log_2{i} \) operations
            \begin{equation*}
            \begin{split}
                & \sum_{i=0}{N} {\log_2{i}} + N\\
                = &\, \log_2{N!} + N\\
                = &\, O(\log{N!})
            \end{split}
            \end{equation*}
    \end{enumerate}
    
\section*{2.15}
    An efficient algorithm in the case would be a binary search, in which the runtime would be at the \( O(\log{N}) \) level.
    
\section*{2.16}
    Code attached as assignment2.cpp.

\end{document}
